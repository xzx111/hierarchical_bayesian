\documentclass[sigconf]{acmart}

\usepackage{booktabs}

\usepackage{bm}
\usepackage{colortbl}
\definecolor{mygray}{gray}{.9}
\usepackage{mathtools}



% Copyright
%\setcopyright{acmcopyright}
%\setcopyright{acmlicensed}
\setcopyright{none}


\begin{document}
\title{Hierarchical Bayesian Personalized Recommendation: A Case Study and Beyond}
%\titlenote{Produces the permission block, and
%  copyright information}
%\subtitle{Extended Abstract}
%\subtitlenote{The full version of the author's guide is available as
%  \texttt{acmart.pdf} document}
%
\author{Author}
\affiliation{
  \institution{Company}
  \streetaddress{Address}
  \city{City} 
  \state{State} 
  \country{Country} 
  \postcode{zip code}
}
\email{email}

% \author{Yan Yan}
% \authornote{correspond author}
% \affiliation{%
%   \institution{JD.com American Technologies}
%   \streetaddress{2900 Lakeside Dr.}
%   \city{Santa Clara} 
%   \state{California}
%   \country{USA} 
%   \postcode{95054}
% }
% \email{yan.yan@jd.com}

% \author{Zitao Liu}
% \affiliation{%
%   \institution{Pinterest Inc.}
%   \streetaddress{651 Brannan St.}
%   \city{San Francisco} 
%   \state{California}
%   \country{USA} 
%   \postcode{94107}
% }
% \email{zitaoliu@pinterest.com}

% \author{Wentao Guo}
% \affiliation{%
%   \institution{JD.com Business Growth BU}
%   \streetaddress{6Fl. T-A, Beichen Century Center}
%   \city{Beijing} 
%   \country{China} 
%   \postcode{100101}
% }
% \email{guowentao@jd.com}


% \author{Darlene King}
% \affiliation{%
%   \institution{Univ. of Texas SW Medical Center}
%   \streetaddress{5323 Harry Hines Blvd}
%   \city{Dallas} 
%   \state{Texas} 
%   \country{USA} 
%   \postcode{75390}
% }
% \email{adarlenea@msn.com}

% The default list of authors is too long for headers}
%\renewcommand{\shortauthors}{B. Trovato et al.}

\begin{abstract}
<<<<<<< HEAD
Items in modern recommender systems are often organized in hierarchical structures. These hierarchical structures and the data within them provide valuable information for building personalized recommendation systems. In this paper, we propose a general hierarchical Bayesian learning framework, i.e., \emph{HBayes}, to learning both the structures and associated latent factors. Furthermore, we develop a variational inference algorithm that is able to learn model parameters with fast empirical convergence rate. The proposed HBayes is evaluated on two real-world datasets from different domains. The results demonstrate the benefits of our approach on item recommendation tasks, and show that it can outperform the state-of-the-art models in terms of both F1 measurement and normalized discounted cumulative gain.
=======
>>>>>>> 0686c90aca7a1d24b696cc404d4777332c85f203

\end{abstract}

\keywords{Hierarchical Bayesian;  Recommendation; Personalization}

\maketitle
% \blfootnote{$^\star$Corresponding author}

%
% The code below should be generated by the tool at
% http://dl.acm.org/ccs.cfm
% Please copy and paste the code instead of the example below.
%
%\begin{CCSXML}
%<ccs2012>
% <concept>
%  <concept_id>10010520.10010553.10010562</concept_id>
%  <concept_desc>Computer systems organization~Embedded systems</concept_desc>
%  <concept_significance>500</concept_significance>
% </concept>
% <concept>
%  <concept_id>10010520.10010575.10010755</concept_id>
%  <concept_desc>Computer systems organization~Redundancy</concept_desc>
%  <concept_significance>300</concept_significance>
% </concept>
% <concept>
%  <concept_id>10010520.10010553.10010554</concept_id>
%  <concept_desc>Computer systems organization~Robotics</concept_desc>
%  <concept_significance>100</concept_significance>
% </concept>
% <concept>
%  <concept_id>10003033.10003083.10003095</concept_id>
%  <concept_desc>Networks~Network reliability</concept_desc>
%  <concept_significance>100</concept_significance>
% </concept>
%</ccs2012>
%\end{CCSXML}
%
%\ccsdesc[500]{Computer systems organization~Embedded systems}
%\ccsdesc[300]{Computer systems organization~Redundancy}
%\ccsdesc{Computer systems organization~Robotics}
%\ccsdesc[100]{Networks~Network reliability}

% We no longer use \terms command
%\terms{Theory}

%\input{samplebody-conf}



\section{Introduction}
\label{sec:intro}
<<<<<<< HEAD
Real-world organizations in business domains operate in a multi-item, multi-level environment. Items and their corresponding information collected by these organizations often reflect a hierarchical structure. For examples, products in retail stores are usually stored in hierarchical inventories. News on web pages is created and placed hierarchically in most websites. These hierarchical structures and the data within them provide a large amount of information when building effective recommendation systems. Especially in e-commerce domain, all products are displayed in a site-wide hierarchical catalog and how to build an accurate recommendation engine on top of it becomes one of the keys to majority companies' business success nowadays. 

However, how to utilize the rich information behind hierarchical structures to make personalized and accurate product recommendations still remains challenging due to the unique characteristics of hierarchical structures and the modeling trade-offs arising from them. Briefly, most well-established recommendation algorithms cannot naturally take hierarchical structures as additional inputs and flatting hierarchical structures usually doesn't work well. It will not only blow up the entire feature space but introduce noise when training the recommendation models. On the other hand, completely ignoring the hierarchies will lead to recommendation inaccuracies. The most common way to alleviate this problem is to feed every piece of data from the hierarchy into a complex deep neural network and hope the network itself can figure out a way to use the hierarchical knowledge. However, such approaches usually behave more like black boxes. They are difficult to debug and cannot provide any interpretation of their intermediate results and outcomes.
=======

>>>>>>> 0686c90aca7a1d24b696cc404d4777332c85f203

In this work, we propose and develop a hierarchical Bayesian, a.k.a., \emph{HBayes}, modeling framework that is able to flexibly capture various relations between items in hierarchical structures from different recommendation scenarios. By introducing latent variables, all hierarchical structures are encoded as conditional independence in HBayes graphical models. Moreover, we develop a variational inference algorithm for learning parameters of HBayes. 

<<<<<<< HEAD
To illustrate the modeling power of the proposed HBayes approach, we introduce HBayes by using a real-world apparel recommendation problem as an example. As an illustration, we consider apparel styles, product brands and apparel items,  and form them into a three-level hierarchical structure. We add additional latent variables as the apparel style membership variables to capture the diverse and hidden style properties of each brand. Furthermore, we include user-specific features into HBayes and extend HBayes into the supervised learning setting where users feedback actions such as clicks and conversions are incorporated. Please note that our HBayes framework is not only limited to apparel recommendation and at the end, we show its flexibility and effectiveness on a music recommendation problem as well.
=======
>>>>>>> 0686c90aca7a1d24b696cc404d4777332c85f203

Overall this paper makes the following contributions:

\begin{itemize}
\item It presents a generalized hierarchical Bayesian learning framework to learn models from rich data with hierarchies.
\item It provides a variational inference algorithm that can learn the model parameters with few iterations.
\item We evaluate our HBayes and its benefits comprehensively in tasks of apparel recommendation on a real-world data set. 
\item We test our framework in different recommendation scenarios to show its generalization and applicability.
\end{itemize}

The remainder of the paper is organized as follows: Section \ref{sec:related} provides a review of existing recommendation algorithms and their extensions in hierarchal learning settings. Section \ref{sec:method} introduces the notations and our generalized HBayes learning framework and its variational inference algorithm. In Section \ref{sec:experiment}, we conduct experiments in a real-world e-commerce data set to show the effectiveness of our proposed recommendation algorithm in different aspects. In addition, we test our model on a music recommendation data set to illustrate the generalization and extended ability of HBayes. We summarize our work and outline potential future extensions in Section \ref{sec:conclusion}.




\section{Related Work}
\label{sec:related}
\subsection{Recommender Systems}

With the explosive growth of digital information over the Internet, the recommender system is considered to be one of the most effective approaches to overcome such information overload. Traditional recommendation algorithms such as item-based approaches learn interactions between users and items and recommend items to users who share similar historical behaviors: collaborative filtering \cite{Sarwar:2001:ICF:371920.372071,Su:2009:SCF:1592474.1722966} and matrix factorization \cite{Rendle:2010:FPM} are both effective approaches under this category.  Content-based approaches including \cite{2011rsh..book...73L,Liu:2011,Yuan:2015} take use of the auxiliary information of both users and items in the content spaces to recommend similar items.  Furthermore, session based recommender systems are developed by analyzing user session information and user visiting patterns \cite{Gultekin_acollaborative,Tang_review:2013}. These particular recommendation techniques take time as an additional dimension, which aims at explicitly modeling user interests over time slices.  Koren \cite{Koren:2010} develops a collaborative filtering type of approach with predictions from static average values combining with a dynamic changing factor. Yin et.al. \cite{Yin:2011} proposes a user-tag-specific temporal interest model to track user interests over time by maximizing the time weighted data likelihood.  For better personalization, Guy et.al. \cite{Guy:2009} proposes an item-based recommender system to utilize each user's social network information. 

\subsection{Bayesian Approaches for Recommendation}

Recently, there are works using Bayesian inferencing for recommender systems.  Rendle et.al. \cite{rendle2009bpr} combines the Bayesian inference and matrix factorization together for learning users implicit feedbacks (click \& purchase) that is able to directly optimize ranking results. Ben-Elazar et.al. \cite{Ben-Elazar:2017} and Zhang et.al. \cite{zhang2007efficient} take user preference consistency into account and develop a variational Bayesian personalized ranking model for better music recommendation.  However, none of these approaches leverage the item structural information when learning each Bayesian models.  Given the fact that the hierarchical structural information widely exist in real-world recommendation scenarios such as e-commerce, social network, music, etc. failing to utilize such information typically makes the Bayesian approaches inefficient and ineffective.  

\subsection{Hierarchical Recommender Systems}

Hierarchical information for recommender systems, as mentioned, is a natural yet powerful entity structure that encodes human knowledge by means of tree based dependency constraints between entities defined by a group of hyper parameters. %and the corresponding hyper-parameters for recommender systems.  
Recommender system hierarchies, in particular, could be either explicit or implicit; either approximate or exact.  There are a few recommendation approaches take use of such information in order to promote items to users who have explicitly visited hierarchically related items or have shown preferences to items that belong to the same sub-categories.  In social tagging systems, Shepitsen et.al. \cite{shepitsen2008personalized} relies on the hierarchies generated by user-taggings, a.k.a folksonomies to build a better personalized recommender system.  In e-commerce, Wang et.al. \cite{wang2018exploring} introduces a hierarchical matrix factorization approach that exploits the intrinsic hierarchical information to alleviate cold start and data sparsity problems.  Despite the fact that these hierarchical recommender systems have received some success, there are still challenges such like: 1. how to infer the hierarchical structure efficiently and accurately if it is not completely explicit; 2. how to better understand the hierarchical topologies discovered by recommendation approaches; 3. how to utilize the hierarchical information for better data explanation all remain unsolved issues.

In this work, we focus on building a generic recommender system that combines the hierarchical structural information as well as the Bayesian inference to better tackle the recommendation problem.  Bayesian learning scheme takes account of both the user preferences and the entity hidden structural properties. It is able to recommend items that align to users' long term tastes.  Meanwhile, the learned latent variables can be well interpreted and explained, which grants us the capability to better analyze the user interests and preferences.

\section{Methodology}
\label{sec:method}
In this work, we develop our generalized hierarchical Bayesian modeling framework that is able to capture the hierarchical structural relations and latent relations in the real-world recommendation scenarios. In our framework, we incorporate both user specific features and user actions into model learning so that more personalized recommendation results can be achieved. Furthermore, we present a variational inference algorithm for the HBayes framework to provide fast learning convergence.

In the following, we will describe how HBayes works by using apparel recommendation as an illustration scenario. Using a real-world example helps explain the hierarchy structural relations and latent variable conditional independence assumptions implied in HBayes. Please note that even we explain HBayes in apparel recommendation, the framework itself can be generalized into other recommendation scenarios with little modification. 



\subsection{Generative Process}

In the real-world scenario, each item or product has to come with a brand and a brand may have more than one items in the hierarchical structures. Therefore, we denote each event $t$ as a 4-tuple (Item, Brand, User, IsClick), i.e, ($\mathbf{X}_t, b_t, u_t, y_t$). $\mathbf{X}_t$ represents the item features associated with event $t$ and $y_t$ is the binary label that indicates whether user $u_t$ has clicked $\mathbf{X}_t$ or not. $b_t$ is the brand of item $\mathbf{X}_t$. 

Furthermore, we expand the hierarchy by a hidden factor, i.e., ``style''. Products from each brand $b_t$ tends to exhibit different styles or tastes, which are unknown but exsit. In this paper, brands are represented as random mixtures over latent styles, where each style is characterized by a distribution over all the items. Let $S$, $B$, $U$ and $N$ be the total number of styles, brands, users and events.

The generative process of HBayes can be described as follows:

\begin{enumerate}
\item[Step 1.] Draw a multivariate Gaussian prior for each style $j$, i.e, $\mathbf{S}_j \sim \mathcal{N}(\mathbf{w}, \delta_s^{-1}\mathbf{I})$ where $j \in \{1, \cdots, S\}$.

\item[Step 2.] Draw a multivariate Gaussian prior for each user $k$, i.e, $\mathbf{U}_k \sim \mathcal{N}(\mathbf{0}, \delta_u^{-1}\mathbf{I})$ where $k \in \{1, \cdots, U\}$.

\item[Step 3.] Draw a style proportion distribution $\boldsymbol{\theta}$ for each brand $i$, $\boldsymbol{\theta} \sim \mbox{Dir}(\boldsymbol{\gamma})$  where $i \in \{1, \cdots, B\}$.

\item[Step 4.] For each brand $i$:
	\begin{enumerate}
		\item[Step 4.1] Draw style assignment $z_{i}$ for brand $i$, $z_{i} \sim \mbox{Mult}(\boldsymbol{\theta})$.
		\item[Step 4.2] Draw $\mathbf{B}_i \sim \mathcal{N}(\mathbf{S}_{z_i}, \delta_b)$.
	\end{enumerate}

\item[Step 5.] For each event $t$, draw $y_t$ from Bernoulli distribution where the probability $p$ is defined as $p(y_t|\mathbf{x}_t, \mathbf{B}_{b_t}, \mathbf{U}_{u_t})$.
\end{enumerate}

\noindent where $\delta_s$, $\delta_u$ and $\delta_b$ are the scalar precision parameters and $\mathbf{w}$ is the prior mean of $\mathbf{S}_j$. 

In this work, in order to have the higher modeling flexibility and capacity, we also treat each distribution's parameter as a random variable and define hyper-priors over them. More specifically, We draw the prior mean $\mathbf{w}$ from $\mathcal{N}(\mathbf{0}, \delta_w^{-1}\mathbf{I})$. For $\delta_w$, $\delta_s$, $\delta_u$ and $\delta_b$, we define Gamma priors over them as follows: 

\begin{equation}
p(\delta_*) = \mathcal{G}(\alpha, \beta)
\end{equation}

\noindent where $\delta_* \in \{ \delta_w, \delta_s, \delta_u, \delta_b \}$. 

The family of probability distributions corresponding to this generative process is depicted as a graphical model in Figure \ref{fig:model}.

\begin{figure}[htb]
\includegraphics[width=0.8\linewidth]{fig/model}
\caption{A graphical model representation of HBayes.}
\label{fig:model}
\end{figure}

\subsection{Probability Priors \& Models}

In this work, we model the probability of a single event $t$ given ($\mathbf{X}_t, b_t, u_t, y_t$) as 

\begin{equation}
p\big(y_t|\bm{X}_t,\bm{B}_{b_t},\bm{U}_{u_t} \big)= \Big[\sigma\big(h_t\big)\Big]^{y_t} \cdot \Big[1-\sigma\big(h_t\big)\Big]^{1-y_t}
\end{equation}

\noindent where $\sigma(\cdot)$ is a logistic function, i.e. $\sigma(x)=(1+e^{-x})^{-1}$. $h_{t}\overset{\mathrm{def}}=\bm{X}_t^T(\bm{B}_{b_t}+\bm{U}_{u_t})$. $\mathbf{X}^T$ is the vector transpose of $\mathbf{X}$. $\bm{U}_{u_t}$ represents user specific information encoded in HBayes for user $u_t$ and $\bm{B}_{b_t}$ denotes the brand $b_t$'s specific information.

Therefore, the conditional likelihood of entire event data $\mathcal{D}$ can be represented as follows:

\begin{equation}
\mathcal{L} = \prod_{t=1}^N p\big(y_t|\bm{X}_t,\bm{B}_{b_t},\bm{U}_{u_t} \big)
\end{equation}

As mentioned in Step 3 of the generative process of HBayes, each brand $i$'s style proportion distribution $\boldsymbol{\theta}$ follows a Dirchlet distribution: $p(\boldsymbol{\theta}) \sim Dir(\boldsymbol{\gamma})$, which is defined as follows: 

\begin{equation}
Dir(\bm{\theta}|\bm{\gamma})=\frac{\Gamma(\sum_{j=1}^{S}\theta_j)}{\prod_{j=1}^{S}\Gamma(\theta_j)}\prod_{j=1}^S \theta_j^{\gamma_j-1}
\end{equation}

\noindent where $\bm{\gamma}$ is the $S$-dimensional Dirichlet hyper-parameter. We initialize $\gamma_j$ by $\frac{1}{S}$. 

Furthermore, in Step 4 of the generative process of HBayes, a brand is modeled as a random mixture over latent styles. Hence, we model the brand parameters by a mixture of multivariate Gaussian distribution defined as follows:

\begin{equation}
p(\bm{B}_i|\bm{z}_i,\bm{S}_{z_i},\delta_b) = \prod_{j}^S p(\bm{B}_i|\bm{S}_j,\delta_b)^{\mathbb{I}(\bm{z}_i=1)} = \prod_{j}^S \mathcal{N}(\bm{B}_i; \bm{S}_j,\delta_b) ^{\mathbb{I}(\bm{z}_i=1)}
\end{equation}

\noindent where $\mathbb{I}(\xi)$ is an indicator function that $\mathbb{I}(\xi)=1$ if the statement $\xi$ is true; $\mathbb{I}(\xi)=0$ otherwise.

Therefore, the log joint likelihood of the dataset $\mathcal{D}$, latent variable $\bm{Z}$ and the parameter $\Theta$ by given $\mathcal{H}$ could be written as follows:

\begin{align}
 & \log \big( p(\mathcal{D},\bm{Z},\Theta|\mathcal{H}) \big) \\
= & \sum_{t=1}^N \log p(y_t|\bm{X}_t,\bm{B}_{b_t},\bm{U}_{u_t})  \\ 
+ &  \sum_{i=1}^B \log p(\bm{B}_i|\bm{z}_i,\bm{S}_{z_i},\delta_b) + \sum_{i=1}^B \log p(\bm{z}_{i}|\bm{\theta}) + \log p(\bm{\theta}|\bm{\gamma})  \\
+ & \sum_{j=1}^S \log p(\bm{S}_j|\bm{w},\delta_s) + \log p(\bm{w}|\delta_w)  \\
+ & \sum_{k}^U \log p(\bm{U}_k|\delta_u)   \\
+ & \log p(\delta_u) + \log p(\delta_b) + \log p(\delta_w) + \log p(\delta_s)
\label{joint_p}
\end{align}

We use $\Theta$ to denote all model parameters:

\begin{equation}
\Theta \overset{\mathrm{def}}= \Big\{\{\bm{U}_k\}, \{\bm{B}_i\}, \{\bm{S}_j\}, \bm{w}, \boldsymbol{\theta}, \delta_u,\delta_b,\delta_s,\delta_w \Big\},
\end{equation}

\noindent where $k \in \{1, \cdots, U\}$, $i \in \{1, \cdots, B\}$, $j \in \{1, \cdots, S\}$ and $\mathcal{H}$ to denote other hyper-parameters: $\mathcal{H} = \{\gamma, \alpha,\beta\}$.


\subsection{Optimization}

\subsubsection{Inference}

\begin{equation}
p(\bm{Z},\Theta|\mathcal{D},\mathcal{H}) = \frac{p(\mathcal{D},\bm{Z},\Theta|\mathcal{H})}{p(\mathcal{D}|\mathcal{H})}
\end{equation}

In this section, we apply variational Bayes to approximate the posterior distribution $p(\bm{Z},\Theta|\mathcal{D},\mathcal{H})$ with a variational distribution $q(\bm{Z},\Theta)$, by maximizing the variational free energy defined as:
\begin{equation}
\mathcal{L}(q)= \int_\Theta \sum_{\bm{Z}} q(\bm{Z},\Theta) \log\frac{p(\mathcal{D},\bm{Z},\Theta|\mathcal{H})}{q(\bm{Z},\Theta)}d\Theta
\end{equation}

\subsubsection{Sigmoid Approximation}

\begin{equation}
\sigma(h_{i}) \geq \sigma(\xi_{i})\exp\big\{\frac{1}{2}(h_{i}-\xi_{i})-\lambda_{i}(h_{i}^2-\xi_{i}^2)\big\}
\end{equation}

$$\lambda_{i}\overset{\mathrm{def}}=\frac{1}{2\xi_{i}}[\sigma(\xi_{i})-\frac{1}{2}]$$

\noindent where $\xi_{i}$ is a variational parameter. This lower bound is derived using the convex inequality. The similar problem was discussed in \cite{jaakkola1997variational,jordan1999introduction}.

\begin{align}
& \big[\sigma(h_{i})\big]^{y_t} \cdot  \big[1-\sigma(h_{i}) \big]^{1-y_t} \\
= & \exp\big\{ y_t h_t \big\} \sigma(-h_{i}) \\
\geq & \sigma(\xi_{i})\exp\big\{y_t h_{i}-\frac{1}{2}(h_{i}+\xi_{i})-\lambda_{i}(h_{i}^2-\xi_{i}^2)\big\}
\end{align}

% \begin{equation}
% \begin{split}
% \sigma(h_{i})^{y_t} \cdot & \big[1-\sigma(h_{i})  \big]^{1-y_t} \geq \\
%  & \sigma(\xi_{i})\exp\big\{y_t h_{i}-\frac{1}{2}(h_{i}+\xi_{i})-\lambda_{i}(h_{i}^2-\xi_{i}^2)\big\}
% \label{bound}
% \end{split}
% \end{equation}



\subsubsection{Learning}

\subsection{Prediction}

\subsection{Summary}



\subsection{Relationship with Other Models}

\section{Experiment}
\label{sec:experiment}
In this section, we conduct several experiments on two data sets: (1) the real-world e-commerce apparel data set; (2) the publicly available music data set.  For both data sets, we compare HBayes against several other \emph{state-of-the-art} recommendation methods which are briefly mentioned as follows:

{\noindent\textbf{HSR} \cite{wang2015exploring}: is an item-based recommendation approach that employs a special non-negative matrix factorization for exploring the implicit hierarchical structure of users and items so the user preference towards certain products are better understood.  We adopt HSR  as one baseline. \newline
\textbf{HPF} \cite{gopalan2015scalable}: generates a hierarchical Poisson factorization model for better modeling users' rating towards certain items based upon each one's latent preferences.  Unlike proposed HBayes, HPF does not leverage the content item feature for constructing the hierarchical structure. We adopt HPF as another baseline. \newline
\textbf{SVD++} \cite{mnih2008probabilistic, koren2008factorization}: combines the collaborative filtering approach and latent factor approach, so to provide a more accurate neighboring based recommendation result.  We adopt SVD++ as another baseline. \newline
\textbf{CoClustering} \cite{george2005scalable}: another collaborative filtering type of approach based on weighted co-clustering improvements that simultaneously cluster users and items.  We adopt CoClustering as another baseline. \newline
\textbf{Factorization Machine (FM)} \cite{rendle2010factorization,rendle2012factorization}: combines support vector machines (SVM) with factorization models which takes the advantage of SVM meanwhile overcomes the feature sparsity issues.  In this paper, we adopt LibFM implementation mentioned in \cite{rendle2012factorization} specifically as another baseline. \newline
\textbf{LambdaMART} \cite{burges2010ranknet}: is the boosted tree version of LambdaRank \cite{donmez2009local}, which is based on RankNet \cite{burges2005learning}.  LambdaMART proves to be a very successful approach for ranking as well as recommendation.  We include LambdaMART as another baseline. \newline}

\subsection{Evaluation Metrics}
Throughout the experiments, we compare HBayes against other baselines on the testing held-out data set under the 5-folds cross-validation settings.  For each fold, after fitting the model on the training set, we rank on the test set by each model, and generate the top M samples with maximal ranking scores for recommendation.  Regarding metrics, we adopt precisions and recalls as well as F1-scores for evaluating the product retrieval quality and normalized discounted information gain (NDCG) for the recommendation's ranking quality.  More specifically, precisions, recalls, and F1-scores are defined as follows:

\begin{eqnarray}
\text{Precision@K} & = & \frac{\text{\# of products clicked in top K}}{\text{K}} \nonumber \\
\text{Recall@K}      & = & \frac{\text{\# of products clicked in top K}}{\text{\# of products the user clicked}} \nonumber \\
\text{F1-score@K} & = & 2\cdot\frac{\text{Precision@K} \cdot \text{Recall@K}}{\text{Precision@K} + \text{Recall@K}}  \nonumber
\end{eqnarray}

discounted cumulative gain (DCG) measures the ranking quality based on the item positions from the resulting list, which is defined as follows:
\begin{eqnarray}
\text{DCG@K} & = & \sum_{i=1}^K\frac{r_i}{\log_2(i+1)} =  r_1 + \sum_{i=2}^K\frac{r_i}{\log_2(i+1)}\nonumber
\end{eqnarray}

re-rank all potential products to produce the ideal maximal possible DCG through out first K items, and its corresponding DCG score is called the ideal discounted cumulative gain (IDCG).  By using both DCG and IDCG, the normalized discounted cumulative gain (NDCG) could be defined as follows:

\begin{eqnarray}
\text{NDCG@K} & = & \frac{\text{DCG@K}}{\text{IDCG@K}}\nonumber
\end{eqnarray}

\subsection{Recommendation on Apparel Data}
The first apparel data set is collected from a large e-commerce company.  In this dataset, each sample represents a particular apparel product which is recorded by various features including: categories, titles, and other properties, etc.  Meanwhile, the user click information are also recorded and translated into data labels. Throughout the experiment, positive labels indicate that certain recommended products are clicked by the user, whereas negative samples indicate that the recommended products are skipped by the user which usually implies that the user is `lack of interest' towards that certain item.  By data cleaning and preprocessing of (1) merging duplicated histories; (2) removing users of too few records, the post-processed data set ends up with $\mathbf{895}$ users, $\mathbf{81223}$ products, $\mathbf{5535}$ brands with $\mathbf{380595}$ uniquely observed user-item pairs.  In average, each user has $\mathbf{425}$ products records, ranging from $\mathbf{105}$ to $\mathbf{2048}$, and $\mathbf{61.2\%}$ of the users have fewer than $\mathbf{425}$ product clicking records.  For each item, we encode the popularity and category features into a $\mathbf{20}$ dimensional feature vector; title and product property into a $\mathbf{50}$ dimensional feature vector.  Combining with other features, the total dimension for each sample ends up with $\mathbf{140}$.

\subsubsection{Feature Analysis}
The apparel data are composed of four types of features: (1) product popularity; (2) product category; (3) product title; (4) product properties.  We briefly explain each feature's physical meaning and how we process the data as follows:

\textbf{Product Popularity} (POP): product popularity is a measure of the prevalence of a item in the dataset. In general, consumers have preferences for a particular product during a period of time. This phenomenon is pretty common for apparel products (\emph{e.g.} apparels in certain styles may be of dominant prevalence especially when famous entertainers are wearing them). For a particular product $i$, the popularity is defined as: $\mathrm{\scriptsize POP}_{i} \coloneqq \frac{n_{x_i}}{\mathcal{N}_\mathbf{x}}$, where $n_{x_i}$ are the number of orders or the contribution of gross merchandise volume (GMV) for product $i$,  and $\mathcal{N}_\mathbf{x} = \sum_{\forall x_i}n_{x_i}$ is the summation of $n_{x_i}$ for all such products in the dataset. \newline

\textbf{Product Category} (CID): In e-commerce, items are clustered into different groups based on the alliance/ similarity of the item functionalities and utilities.  In e-commerce websites, such an explicit hierarchy usually exists.  We encode each item's category into a high dimensional vector via one-hot encoding and adjust the feature weights by the popularity of such category. \newline

\textbf{Product Title} (TITLE): product titles are created by vendors and they are typically in the forms of natural languages indicating the product functionality and utility.  Examples could be like `\emph{INMAN short sleeve round neck triple color block stripe T-shirt 2017}'.  We preprocess the product titles by generating the sentence vector embedding based on \cite{de2016representation}.  The main idea is to average the wording weights in the title sentence based on the inverse document frequency (IDF) value of each individual word involved. \newline

\textbf{Product Property Features} (PROP): other product meta data features are also provided in the apparel dataset. For instance, the color feature of items takes value like: `black', `white', `red', etc, and the sizing feature takes value like `S', `M', `L', `XL', etc.  Similar as category features, each product property is first encoded into a binary vector $\bm{x}_i \in \{0, 1\}^{|N|}$, where $N$ denotes the set of all possible corresponding product values.  Then we encode all such property binary vectors into a fixed length ($50$) vector. \newline

On one hand, by utilizing more features HBayes in general reaches better performance in terms of precision-recall metrics.  We report PR-AUC in table (\ref{tab:features_cmp}) to prove this argument; on the other hand, more features needs much more training time.  Figure (\ref{fig:train_time_cmp}) reports the changes in likelihood against training time cost (in minutes) for each feature use cases\footnote{The experiment is conducted via the same Linux Qual core 2.8 GHz Intel Core i7 macbook with 16 Gigabytes of memory}. It is shown that by only taking POP features, model spends less than 5 minutes to converge while taking POP+CID+TITLE+PROP features, the model needs more than 6 hours in training.   

\begin{table}[htb]
\centering
\scalebox{0.85}{
\begin{tabular}{l|c}
\toprule
\textbf{Features} & \textbf{PR AUC} \\
\hline
POP & 0.0406 \\
\rowcolor{mygray}
POP+CID & 0.0414 \\
POP+CID+TITLE & 0.0489 \\
\rowcolor{mygray}
POP+CID+TITLE+PROP & 0.0491 \\
\bottomrule
\end{tabular}}
\caption{Model performance under different feature combinations in terms of PR AUC}
\label{tab:features_cmp}
\end{table}

\begin{figure*}[!htb]
\includegraphics[width=0.7\linewidth]{fig/legend}
\end{figure*}

\begin{figure*}[!htb]
\minipage{0.32\textwidth}
  \includegraphics[width=\linewidth]{fig/F1-score_jd}
  \caption{F1@K on \emph{Apparel} data.}
  \label{fig:perf_cmp_F1_apparel}
\endminipage\hfill
\minipage{0.32\textwidth}
  \includegraphics[width=\linewidth]{fig/precision_jd}
  \caption{Precision@K on \emph{Apparel} data.}
  \label{fig:perf_cmp_precision_apparel}
\endminipage\hfill
\minipage{0.32\textwidth}%
  \includegraphics[width=\linewidth]{fig/recall_jd}
  \caption{Recall@K on \emph{Apparel} data.}
  \label{fig:perf_cmp_recall_apparel}
\endminipage
\end{figure*}


\begin{figure*}[!htb]
\minipage{0.32\textwidth}
  \includegraphics[width=\linewidth]{fig/F1-score_music}
  \caption{F1@K on \emph{Music} data.}
  \label{fig:perf_cmp_F1_music}
\endminipage\hfill
\minipage{0.32\textwidth}
  \includegraphics[width=\linewidth]{fig/precision_music}
  \caption{Precision@K on \emph{Music} data.}
  \label{fig:perf_cmp_precision_music}
\endminipage\hfill
\minipage{0.32\textwidth}%
  \includegraphics[width=\linewidth]{fig/recall_music}
  \caption{Recall@K on \emph{Music} data.}
  \label{fig:perf_cmp_recall_music}
\endminipage
\end{figure*}

\begin{figure}[!htb]
\centering
\minipage{0.49\columnwidth}
  \includegraphics[width=\linewidth]{fig/1feature_lik}
\endminipage
\minipage{0.49\columnwidth}
  \includegraphics[width=\linewidth]{fig/2feature_lik}
\endminipage
\vskip\baselineskip
\minipage{0.49\columnwidth}
  \includegraphics[width=\linewidth]{fig/3feature_lik}
\endminipage
\minipage{0.49\columnwidth}
  \includegraphics[width=\linewidth]{fig/4feature_lik}
\endminipage
\caption{Training time under different feature combinations on e-commerce recommendations}
\label{fig:train_time_cmp}
\end{figure}

%\begin{figure}[htb]
%\includegraphics[width=0.9\columnwidth,height=0.5\columnwidth]{fig/Lik_time}
%\caption{Training time under different feature combinations on e-commerce recommendations}
%\label{fig:train_time_cmp}
%\end{figure}

\begin{table}[htb]
\begin{center}
\scalebox{0.85}{
\begin{tabular}{l|cccc}
\toprule
\textbf{Method} & \textbf{NDCG@5} & \textbf{NDCG@10} & \textbf{NDCG@25} & \textbf{NDCG@50} \\
\hline
\rowcolor{mygray}
CoClustering & 0.1288 & 0.1637 & 0.2365 & 0.3050 \\
NMF & 0.1249 & 0.0156 & 0.2272 & 0.3020 \\
\rowcolor{mygray}
SVD++ & 0.1138 & 0.1487 & 0.2287 & 0.3073\\
HSR & 0.1266 & 0.1603 & 0.2354 & 0.3107 \\
\rowcolor{mygray}
HPF & 0.1412 & 0.1757 & 0.2503 & 0.3229 \\
FM & 0.1363 & 0.1592 & 0.2291 & 0.3117\\
\rowcolor{mygray}
LambdaMART & 0.1287 & 0.1585 & 0.2304 & 0.3123\\
HBayes & \textbf{0.1557} & \textbf{0.1974} & \textbf{0.2871} & \textbf{0.3590}\\
\bottomrule
\end{tabular}}
\end{center}
\caption{NDCG on apparel recommendations}
\label{NDCG_cmp}
\end{table}

\subsubsection{Performance Comparison}
We first report the model performance regarding precision, recall, as well as F1-score for HBayes against other baselines in Figure (\ref{fig:perf_cmp_F1_apparel},\ref{fig:perf_cmp_precision_apparel},\ref{fig:perf_cmp_recall_apparel}).  As shown when each method recommend fewer number of products (K = 5), HBayes does not show the superiority regarding with recalls, with the increment of recommended items, the recall for HBayes becomes much better against others, which implies HBayes is really efficient in terms of finding items that people tend to take interest in.  In the sense of precision, HBayes is consistently better than other baseline methods which implies HBayes is much more accurate in terms of item classification.   Given the performance of precisions and recalls, HBayes is much better regarding F1-score with different K for apparel recommendation.

Regarding the ranking quality, we use NDCG to report each method's performance in Table (\ref{NDCG_cmp}).  HBayes is superior against other baseline methods through out different K recommended.  Specially, HBayes beats the second best HPF at $\text{K}=5$ by $10.3\%$, at $\text{K}=10$ by $12.4\%$, at $\text{K}=25$ by $14.7\%$ and at $\text{K}=50$ by $11.2\%$.  

\subsubsection{Model Learning Analysis}
\begin{figure}
\includegraphics[width=0.58\columnwidth,height=3cm]{fig/brand_tsne}
\includegraphics[width=0.39\columnwidth,height=4cm]{fig/style-brand}
\caption{tSNE of four latent apparel style clusters}
\label{fig:tsne-represntation-style-cluster}
\end{figure}

Like mentioned in Sec.\ref{sec:method}, HBayes learns the latent style clusters, and group different brands of products based on their different hidden style representations.  Figure (\ref{fig:tsne-represntation-style-cluster}) shows the tSNE \cite{maaten2008visualizing} representations for different apparel clusters learned by HBayes and we randomly pick $4$ samples out of each cluster and display their product images at the right subfigure.  As shown, cluster one taking the majority proportion of apparel items seems about stylish female youth garment.  This intuitively makes sense since the majority of apparel customers are female youth for e-commerce websites; as a result, most apparels are also focusing on female customer audience.    The second cluster seems about senior customers who are elder in age.   Interestingly, the third cluster and the fourth cluster which are closely tied up are both about male customer youth.  However, the third cluster seems focusing more on office business garment while the fourth cluster seems more about K-pop street styles.  This indicates us that the HBayes indeed learns the meaningful intrinsic garment styles from apparel items by leveraging customer behavior data.   

\subsection{Recommendation on Last.fm Music Data}
The second data set is collected from Last.fm dataset \cite{Celma:Springer2010} and Free Music Archive (FMA) \cite{FMA}. Last.fm is a publicly available dataset which contains the whole listening habits (till May, 5th 2009) for $\mathbf{1000}$ users. FMA is an open and easily accessible dataset providing $\mathbf{917}$ GiB and $\mathbf{343}$ days of Creative Commons-licensed audio from $\mathbf{106574}$ tracks, $\mathbf{16341}$ artists and $\mathbf{14854}$ albums, arranged in a hierarchical taxonomy of $\mathbf{161}$ genres. It also provides full-length and high quality audios with precomputed features.  In our experiment, tracks in Last.fm dataset were further intersected with FMA dataset for better feature generation.  The resulting dataset contains $\mathbf{500}$ users, $\mathbf{16328}$ tracks and $\mathbf{36}$ genres.

\subsubsection{Performance Comparison}
We conduct similar experiments as we do for apparel data set and report precisions, recalls as well as F1-scores in Figure (\ref{fig:perf_cmp_F1_music},\ref{fig:perf_cmp_precision_music},\ref{fig:perf_cmp_recall_music}).  Although HPF and HBayes share similar performance regarding recalls along with different K, HBayes is dominant for precision at different K, specially when K is small (5, 10), which indicates HBayes is very efficient and precise for helping users pick up the songs they prefer even when the recommended item lists are short.   Combining the two, HBayes is superior in F1-scores for different K recommended.  

In terms of ranking quality, we report the NDCG performance in Table (\ref{NDCG_cmp_music}).  Similar as e-commerce apparel data, HBayes is the best approach and HPF is the second best one throughout different K items recommended.  Specifically, HBayes beats HPF for $2.7\%$ at $\text{K}=5$, $4.3\%$ at $\text{K}=10$, $2.1\%$ at $\text{K}=25$, and $2.5\%$ at $\text{K}=50$ separately.

\begin{table}[htb]
\begin{center}
\scalebox{0.85}{
\begin{tabular}{l|cccc}
\toprule
\textbf{Method} & \textbf{NDCG@5} & \textbf{NDCG@10} & \textbf{NDCG@25} & \textbf{NDCG@50} \\
\hline
\rowcolor{mygray}
CoClustering & 0.2415 & 0.2314 & 0.2289 & 0.2349 \\
NMF & 0.2556 & 0.2494 & 0.2368 & 0.2431 \\
\rowcolor{mygray}
SVD++ & 0.2493 & 0.2478 & 0.2381 & 0.2439\\
HSR & 0.2544 & 0.2495 & 0.2384 & 0.2448\\
\rowcolor{mygray}
HPF & 0.2584 & 0.2513 & 0.2405 & 0.2474\\
FM & 0.2527 & 0.2453 & 0.2284 & 0.2333\\
\rowcolor{mygray}
LambdaMART & 0.2372 & 0.2337 & 0.2272 & 0.2218\\
HBayes & \textbf{0.2655} & \textbf{0.2620} & \textbf{0.2455} & \textbf{0.2537}\\
\bottomrule
\end{tabular}}
\caption{NDCG on Last.fm recommendations}
\label{NDCG_cmp_music}
\end{center}
\end{table}

%For PR-AUC, HBayes (0.055) again beats against the second best, which is LambdaMART (0.043) by $27.9\%$.  In terms of F1-score, HBayes in average beats against LambdaMART by $14.0\%$.  Regarding with the ranking quality, HBayes beats against the second best, which is NMF  for $M=5,10,25$ by $4.20\%$ in average and SVD++ for $M=50$ by $4.02\%$.  











\section{Conclusion(Chris)}
\label{sec:conclusion}


% \balance
\bibliographystyle{ACM-Reference-Format}
\bibliography{recsys2018}

\end{document}
