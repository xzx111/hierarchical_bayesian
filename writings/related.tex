\subsection{Recommender Systems}

<<<<<<< HEAD
With the explosive growth of digital information over the Internet, the recommender system is considered to be one of the most effective approaches to overcome such information overload. Traditional recommendation algorithms such as item-based approaches learn the interaction between users and items and recommend items to users who share similar historical behaviors: collaborative filtering \cite{Sarwar:2001:ICF:371920.372071,Su:2009:SCF:1592474.1722966} and matrix factorization \cite{Rendle:2010:FPM} are both effective approaches under this category.  Content-based approaches including \cite{2011rsh..book...73L,Liu:2011,Yuan:2015} take use of the auxiliary information of both users and items in the content spaces so to recommend similar items.  Furthermore, by looking into users session information and analyzing such visiting patterns, there is another branch of recommender system methodologies which are session-based \cite{Gultekin_acollaborative,Tang_review:2013}.  They take \emph{time} as another input dimension besides items and users, which aims at explicitly modeling the user interest over time slices to combine with various classic recommender algorithms. \cite{Koren:2010} develops a collaborative filtering type of approach with predictions from static average value combining with a dynamic changing factor.  \cite{Yin:2011} proposes a user-tag-specific temporal interest model to track the user interest over time by maximizing the time-weighted data likelihood.  For better personalization, \cite{Guy:2009} proposes an item-based recommender system combining with each user's social network information. 
\subsection{Bayesian Approaches for Recommendation}

Recently, there are works using Bayesian inferencing for recommender systems. \cite{rendle2009bpr} combines the Bayesian inference and matrix factorization together for learning users implicit feedbacks (click \& purchase) so to directly optimize the ranking results in the sense of AUC metrics. \cite{Ben-Elazar:2017} and \cite{zhang2007efficient} takes user preference consistency into account and develops a variational Bayesian personalized ranking model for better music recommendation.  However, none of these approaches leverage the entity structural information when learning the Bayesian models.  Given the fact that hierarchical structures widely exist in several recommendation practices such like e-commerce recommender systems (Fig.\ref{XX} shows a typical e-commerce structure); social-network recommender systems; music recommender systems; etc. fails to build in such structural information typically results in those Bayesian approaches not as efficient as they supposed to.  

\subsection{Hierarchical Recommender Systems}
Hierarchical information for recommender systems, as mentioned, is a natural yet powerful entity structure that encodes human knowledge by means of tree-based dependency constraints between items and the corresponding hyper-parameters for recommender systems.  Recommender system hierarchies could be explicit or implicit, and there are quite a few approaches take use of such information in order to recommend items to users who have explicitly visited other hierarchically related items or shown preferences to items that are hierarchically belonging to the same sub-categories.  In social tagging systems, Shepitsen et.al. \cite{shepitsen2008personalized} relies on the hierarchies generated by user-taggings, a.k.a folksonomies to build a better personalized recommender system.  In e-commerce, Wang et.al. \cite{wang2018exploring} introduced a hierarchical matrix factorization approach that exploits the intrinsic hierarchical information to alleviate cold start and data sparsity problems.  Despite the fact that these hierarchical recommender systems have received some success, there are still some challenges such like: 1. how to design the hierarchical structure and learn such structure efficiently if it is not completely explicit; 2. how to better understand the implicit hierarchical topologies discovered by recommendation approaches and ... all remain unsolved issues.

\subsection{Hierarchical Bayesian Inference}
In our work, we focus on building a generic recommender system that combines the hierarchical structural information as well as the Bayesian inference to better tackle the recommendation problem.  Bayesian learning scheme takes account of both the user preferences and product hidden structural properties. It is able to recommend products that align with users' long-term tastes. Meanwhile, the learned latent variables can be well interpreted and explained, which grants us the capability to analyze and understand users interests and preference.
=======

\subsection{Bayesian Approaches for Recommendation}


\subsection{Hierarchical Recommender Systems}


>>>>>>> 0686c90aca7a1d24b696cc404d4777332c85f203
