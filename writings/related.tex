\subsection{Recommender Systems}

With the explosive growth of digital information over the Internet, the recommender system is considered to be one of the most effective approaches to overcome such information overload. Traditional recommendation algorithms such as item-based approaches including collaborative filtering \cite{Sarwar:2001:ICF:371920.372071,Su:2009:SCF:1592474.1722966} and matrix factorization \cite{Rendle:2010:FPM} recommend users different items / entities by learning the interaction between users and items, while content-based approaches including \cite{2011rsh..book...73L,Liu:2011,Yuan:2015} compare the auxiliary information of both users and items to recommend similar items.  Further, by looking into time variant nature of recommender systems, there are time-aware recommendation approaches \cite{Gultekin_acollaborative,Tang_review:2013} that take \emph{time} as another input dimension besides items and users, which aims at explicitly modeling the user interest over time slices to combine with various classic recommender algorithms. \cite{Koren:2010} develops a collaborative filtering type of approach with predictions from static average value combining with a dynamic changing factor.  \cite{Yin:2011} proposes a user-tag-specific temporal interest model to track the user interest over time by maximizing the time weighted data likelihood.  For better personalization, \cite{Guy:2009} proposes an item-based recommender system combining with each user's social network information. 
\subsection{Bayesian Approaches for Recommendation}

Recently, there are a group of works using Bayesian inferencing for recommendations. \cite{rendle2009bpr} combines the Bayesian inference and matrix factorization together and by learning users implicit feedbacks (click \& purchase) to directly optimize the ranking results in the sense of AUC metrics. \cite{Ben-Elazar:2017} and \cite{zhang2007efficient} takes user preference consistency into account and develops a variational Bayesian personalized ranking model for better music recommendation.  However, none of those approaches take the product rich structure into account while learning the Bayesian models.  Given the fact that hierarchical structures widely exist in several use cases including e-commerce (Fig.\ref{XX} shows a typical e-commerce structure), social-network recommendation, music recommendation, etc. ignoring such rich structural infomation typically results in the Bayesian inference suboptimal approaches.  

\subsection{Hierarchical Recommender Systems}
Hierarchy is a natural yet powerful structure that encode human knowledge by means of ... Recently, there are quite a few recommendation approaches take use of this piece of information in order to recommend items that are hierarchically related to the items users have explicitly visited.  In e-commerce, XXX...   In social tagging systems, XXX relies on hierarchies generated by user tags, a.k.a folksonomies to build a better personalized recommender system. ...  In field3, XXX exploits the intrinsic hierarchical information to alleviate cold start and data sparsity problems ...

\subsection{Hierarchical Bayesian Inference}
In our work, we focus on building a generic recommender system that combines the hierarchical structural infomation as well as the Bayesian inference to better tackle the recommendation problem.  Bayesian learning scheme takes account of both the user preferences and product hidden structural properties. It is able to recommend products that align to users' long term tastes. Meanwhile the learned latent variables can be well interpreted and explained, which grants us the capability to analyze and understand users interests and preference.

%we can have better understanding on users' likes and dislikes, so to target certain group of users based on their social behavior more accurate and efficient, which is another very important topic in the advertising business.   
% Despite multiple branches of research regarding recommender system, none of these approaches take into account of hierarchical information existing in products for better structural understanding, which leaves their recommendation results sub-optimal.  
